\documentclass{article}

\usepackage[latin1]{inputenc}
\usepackage[T1]{fontenc}
\usepackage[portuguese]{babel}

\begin{document}

\title{Algoritmos e Estruturas de Dados\\ Gest�o de uma Biblioteca}
\author{Turma 4\\\\Grupo B\and Jos� Peixoto \\Paulo Faria \\Pedro Moura  \and 200603103\\201204965\\201306843}
\maketitle
\newpage
\begin{abstract}
Foi nos proposto o desenvolvimento de uma aplica��o em C++ para gest�o de uma biblioteca que abrangesse livros leitores empregados e supervisores. A biblioteca teria necessariamente que permitir de forma controlada o empr�stimo de livros a leitores e gest�o dos mesmos pelos empregados. Os empr�stimos, quando poss�veis, teriam um per�odo m�ximo de dura��o de 7 dias para cada livro, podendo o leitor incorrer numa multa caso n�o o devolvesse dentro deste prazo. Os supervisores poderiam gerir uma equipa individual composta por empregados.
\end{abstract}

\section{Descri��o da solu��o implementada}
\section{Lista de casos de utiliza��o da aplica��o}
\section{Relato das principais dificuldades encontradas no desenvolvimento do trabalho}
\section{Indica��o do esfor�o dedicado por cada elemento do grupo}
\end{document}

